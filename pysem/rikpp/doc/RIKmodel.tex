\chapter{RIK model}
\label{chapRIK}


RIK (Ruiz's Integral Kinematic) model is an advanced kinematic source model developed by \cite{Ruiz2011} and further modified by \cite{Gallovic2016}. This method allows taking into account frequency-dependent directivity effects. The asperities have a fractal distribution that may follow a given inverted slip distribution. \\

This code is a forked version of the original version provided by F. Gallovic \footnote{\texttt{https://github.com/fgallovic/RIKsrf}~\cite{Gallovic2016}}. The original code version was interfaced with the Spectral Element Method code \texttt{SEM3D}~\cite{SEM3D}. The kinematic source model could be generated by using the \texttt{RIKsrf} code, compiled with gfortran or ifort, which generate the executable \texttt{RIKsrf2}. \\ The \texttt{python3} API \texttt{rikpp} is attached to the code to post-process the output files. 

\newpage 

\section{Installation}

In order to retrieve the code, and compile it the following commands need to be executed:
\begin{enumerate}
    \item Clone the repository to the local directory of installation
    \item[] \texttt{git clone https://github.com/FilLTP89/RIKsrf.git}
    \item Enter the downloaded \texttt{RIKsrf} directory
    \item[] \texttt{cd \texttt{RIKsrf}}
    \item Switch current branch to filippo brach
    \item[] \texttt{git checkout -b filippo}
    \item[] \texttt{git branch --set-upstream-to=origin/filippo filippo}
    \item Fetch from the remote repository
    \item[] \texttt{git pull}
    \item Compile the code
    \item[] \texttt{cd src-RIKsrf}
    \item[] \texttt{. compile\_gfortran.sh}\footnote{If ifort is available, the code can be alternatively compiled with \texttt{. compile\_ifort.sh}}
\end{enumerate}


\section{Example}


\texttt{RIKsrf} code works by calling the executable \texttt{RIKsrf2} from any example folder. In this manual, a case from South Napa fault model is used as an example. The related folder to this example is \texttt{examples/Napa}. The input files needed for the execution of the \texttt{RIKsrf} code are summarized below :
\begin{enumerate}
    \item \texttt{RIKsrf.in}: Contains the description of the extended fault.
    \item \texttt{Napaslipfrominversion.dat}: Distribution of the several sub-sources on the fault plane for Napa case.
    \item \texttt{crustal.dat} : Description of the velocities of layers in the model.
\end{enumerate} 

\subsection{Generation of the slip distribution - \texttt{RIKsrf}}
Once those files are prepared the code can be executed and provide the necessary slip distribution for the extended fault. The execution of the code is as follows:
\begin{itemize}
    \item Go to the directory of the Napa example
    \item[] \texttt{cd \texttt{RIKsrf}/examples/Napa}
    \item Execute the code
    \item[] \texttt{echo RIKsrf.in $|$ ../../src-RIKsrf/RIKsrf2}
\end{itemize}

After the simulation, the code sorts a number of output files such as slipdistribution.dat, subsources.dat etc. 

\subsection{Conversion of slip data to be used in \texttt{SEM3D}}
In order to exploit the fault generation and use it in \texttt{SEM3D} a post-treatment process needs to be followed so as to re-orient and re-position the generated fault in the coordinates of the \texttt{SEM3D} model. For this purpose we are going to exploit the RIK post-processing routines found in sem-ecp/pysem/rikpp:
\begin{itemize}
    \item Convert the slip distribution to moment distribution (used by \texttt{SEM3D})
    \item[] \texttt{python3 rik\_slip2moment.py @nL 150 @nW 100 @sf ./slipdistribution.dat @hL 12.5 @hW 0.0 @L 15.0 @W 10.0 @fg napa @strike 45. @dip 82. @rake 108. @nt 480 @dt 0.025 @hE 631892.2 @hN 4931475.0 @hZ -860.0 @wkd ./ @tag napa}
\end{itemize}

where
\begin{itemize}
    \item \texttt{L} and \texttt{W} = Length and Width of the fault
    \item \texttt{nL} and \texttt{nW} = number of subsources along \texttt{L} and \texttt{W}
    \item \texttt{hL} and \texttt{hW} = hypocenter along \texttt{L} and \texttt{W}
    \item \texttt{hE} and \texttt{hN} = position of the hypocenter on the mesh
    \item \texttt{nt} and \texttt{dt} = number of time-steps and the value of the time-step
    \item \texttt{wkd} = working directory
    \item \texttt{tag} = the name of the output files
    \item \texttt{sf} = the file with the slipdistribution generated from the \texttt{RIKsrf} code
\end{itemize} 

The output of the code provides the files @tagname\_kine.h5 and @tagname\_moment.h5 to be used in \texttt{SEM3D} for the definition of the extended fault.

\subsection{Post processing}
In order to verify the generation of the extended fault it is possible to plot the distribution of the slip via the use of the SEM/sem-ecp/pysem/rikpp/rik\_plot\_slip.py routine:
\begin{itemize}
    \item Slip distribution plot
    \item[] \texttt{python3 /SEM/sem-ecp/pysem/rikpp/rik\_plot\_slip.py @wkd ./ @nL 150 @nW 100 @sf ./slipdistribution.dat @hL 12.5 @hW 0.0 @L 15.0 @W 10.0 @tag napa}
\end{itemize}
Input parameters are the same as previously defined for the conversion of the data and the provided output is a \texttt{@tagname\_slipmax.png} file showing generated slip distribution. The results obtained in this example are presented in Fig.~\ref{fig:slip}.

\begin{center}
\leavevmode
\includegraphics[scale=0.45]{Animation/napa_slipmax.png} 
\captionof{figure}{Maximum-slip distribution on fault plan of Napa event.}
\label{fig:slip} 
\vspace{1cm}
\end{center}
\noindent 
Except the visual representation of the maximum slip on the plane of the fault, it is also possible to obtain the animation with the rupture propagation along the fault plane. This time the command to be used is the SEM/sem-ecp/pysem/rikpp/rik\_plot\_slip.py:
\begin{itemize}
    \item Rupture propagation on the fault plane
    \item[] \texttt{python3 /SEM/sem-ecp/pysem/rikpp/rik\_fault2xmf.py @wkd ./ @mf tagname\_moment.h5 @kf tagname\_kine.h5}
\end{itemize}
where \texttt{tagname} is the name that was previously given for the generation of the hdf5 files.

The output of the code generates an \texttt{@tagname\_moment.xmf} file that can be visualised in \texttt{paraview} and provides the animation of the rupture with respect to time. %Results for the Napa case are presented with Fig.~\ref{fig:animation}.

% \begin{figure}[h]
% \centering
% \animategraphics[autoplay, loop,  scale = 0.19]{15}{Animation/Video_}{0000}{0172}
% %\includegraphics[scale=0.19]{Animation/Video_0000.png} 
% \captionof{figure}{Rupture propagation on the fault plane.}
% \label{fig:animation}
% \vspace{1cm}
% \end{figure}


 